%Validation_OhmsLaw
\chapter{Validation of Ohm's Law}
\vspace{-0.25in}
What physicists do is to try to understand how the universe works. To do this we use the Scientific Method. And what makes the scientific method different from philosophy is the use of experimentation to verify our ideas.

So in a physics lab class, we need to test ideas about how the universe works. We call these ideas ``mental models'' or just ``models.'' We have been using one of these models in making voltage measuring devices already. It was called Ohm's law. Let's start out by testing Ohm's law to see if it really works.

\section{Ohm's Law Revisited}

We learned several labs ago that voltage and current are linearly related to each other. This is what we would call a \emph{model}, a mental understanding of how part of the universe works. Usually in physics we distil the model into an equation . We call this equation a law. In this case, Ohm's law. 

\begin{equation*}
	\Delta V=IR
\end{equation*}

\noindent where $R$ is the slope of the $\Delta V$ vs. $I$ curve. We can see the model relationship between $\Delta V$ and $I$ reflected in the equation. Note that being a ``law'' doesn't mean the equation is always true. The word ``law'' generally implies that the equation is true at least some of the time, but really it is telling us we have distilled our model into math.

We might plot our equation to show the $\Delta V$ vs. $I$ relationship.

\begin{figure}[h!]
	\centering
	\includegraphics[width=2.0in,height=1.4in]{PH4CAV2O}
\end{figure}



\noindent But as scientists, we should ask, does our model work for all materials? What if we graphed $\Delta V$ vs $I$ for some device and found a graph that looks like this?

\begin{figure}[h!]
	\centering
	\includegraphics[width=2.2606in,height=1.5177in]{PH4CAV2P}
\end{figure}

\noindent Such a device would \emph{not} follow Ohm's law. We would say that such a device is \emph{nonohmic}.

Today we will test our model by taking $\Delta V$ and $I\ $measurements and
seeing if the equation $\Delta V=IR$ describes the data well. Of course,
this means we need to measure two things at once with our Arduino. We need
both $\Delta V$ and $I.$ But this isn't a problem because our Arduninos have five analog inputs. So we just need to have one measurement attached to, say, pin A0 and another to, say, pin A1. Of course both will need to be
connected to GND as the second measurement because $\Delta V$ measurements
take two leads. 

But wait, if we are testing Ohm's law, we don't want two $\Delta V$
measurements, we want $\Delta V$ and $I.$ How do we get $I$ measured by an
Arduino?

\subsection{Measuring current with our Arduino}

Arduinos and other DAQs only measure voltages. Let's review how we measure
the voltage across a resistor, and then review how to turn that voltage
measurement into a current measurement.

\begin{figure}[h!]
	\centering
	\includegraphics[width=3.2785in,height=1.6449in]{PH4CAV2Q}
\end{figure}

We put the two leads of a voltmeter (shown as a circle with a $\Delta V$\ in it) on either side of the resistor that we are testing. If the voltmeter is our Arduino, the leads on the side of the resister connected to the positive side of the battery should go to A0 and the lead connected to the negative side of the battery should be connected to GND. This is all what an Arduino (or any other DAQ) can do.

To measure a current with our Arduino we have to somehow turn that current
into a voltage. This is true of most measurements we will do. We need to
turn temperature, or humidity, or magnetic field, or light intensity into a
voltage. Turning magnetic field into a voltage is a little tricky, but we
already know all we need to know to handle current.

To turn a current into a voltage, think of Ohm's law again.

\begin{equation*}
	\Delta V_{s}=IR_{s}
\end{equation*}

\noindent we can solve for $I$

\begin{equation*}
	I=\frac{\Delta V_{s}}{R_{s}}
\end{equation*}

\noindent so if we add a new resistor, $R_{s}$ to the circuit,

\begin{figure}[h!]
	\includegraphics[width=3.7395in,height=2.7423in]{PH4CAV2R}
\end{figure}

\noindent and measure the voltage across that circuit, we will be able to calculate the current.

Of course, if $R_{s}$ is very large, then $R_{s},$ itself, will slow down
the current. So we want to choose a $R_{s}$ that is much less than $R_{test}. $

\begin{equation*}
	R_{s}\ll R_{test}
\end{equation*}

\noindent But so long as this is true, our $R_{s}$ won't change the current much, and since we know $R_{s}$ we know the current 

\begin{equation*}
	I=\frac{\Delta V_{s}}{R_{s}}
\end{equation*}

\noindent We have turned our current measurement into a voltage measurement!

\subsection{Actually making an Arduino measure current}

This idea is great, but let's talk a little bit about how to wire this dual
measurement. Think again about our two voltage measurements, $\Delta V$ and $\Delta V_{s}.$ Each $\Delta $ implies two measurements. That means we need a total of four measurements to make this work! Let's see where these
measurements would be on our circuit diagram.

\begin{figure}[h!]
	\centering
	\includegraphics[width=4.2583in,height=3.1185in]{PH4CAV2S}
\end{figure}

By drawing the diagram, we realize that we can create both $\Delta V$ measurements with only three individual voltage measurements because the voltage at point 2 and the voltage at point 3 should be the same. So we could wire our circuit like this:

\begin{figure}[h!]
	\includegraphics[width=4.2004in,height=2.8928in]{PH4CAV2T}
\end{figure}

\noindent and of course we need to wire the negative pole of the battery or power supply to the GND pin. Then our two voltage difference measurements will be formed from 

\begin{eqnarray*}
	\Delta V &=&V_{A2}-V_{A1} \\
	\Delta V_{s} &=&V_{A1}-V_{A0}
\end{eqnarray*}

If we keep our voltage from our battery or power supply in the $0\unit{V}$
to $+5\unit{V}$ range, then we can use our simple voltmeter sketch. We do
need to modify it to take three different voltage measurements. And we need
to add the math to make $\Delta V_{s}$ into $I.$ We could even modify this
so that our code would report out 

\begin{equation*}
	R=\frac{\Delta V}{I}
\end{equation*}

\noindent and we might as well. Here is an example sketch.

%%%%%%%%%%%%%%%%%%%%%%%%%%%%%%%%%%%%%%%%%%%%%%%%%%%%%%%%%%%%%%%%%%%%%%%%
% code link here for download
%\href{https://dtoliphant.github.io/PH250Manual/Code/OhmsLaw_SimpleVA.ino}{Download here}
%\href{https://dtoliphant.github.io/PH250Manual/Code/DAQ_Extended_voltmeter.ino}{Download here}
\href{https://raw.githubusercontent.com/rtlines/IntermediateLabPH250/main/Code/DAQ_Extended_voltmeter.ino}{Download here}
%%%%%%%%%%%%%%%%%%%%%%%%%%%%%%%%%%%%%%%%%%%%%%%%%%%%%%%%%%%%%%%%%%%%%%%%
\lstinputlisting[language=Arduino]{Code/OhmsLaw_SimpleVA.ino}

\subsubsection{Choosing shunt resistors}

It's harder than you might think to choose a good shunt resistor. The shunt
resistor resistance shouldn't be big enough to cause too much error in our $\Delta V$ measurement for the resistor we want to measure. Nor should it be so large that it effects the current much. But suppose we find the smallest resistor that we can, say, $10\unit{\Omega}.$ Surely that will not affect the actual $\Delta V$ or $I$ measurements. But still, we may have a problem. Let's consider an actual circuit to see
why.

Suppose we have a circuit where the input voltage from the power supply is $\Delta V_{ps}=2\unit{V}$ and our test resistor is $R=5000\unit{\Omega}.$ And suppose we try to use $R_{s}=10\unit{\Omega}.$ 

\begin{figure}[h!]
	\includegraphics[width=4.6527in,height=2.9265in]{PH4CAV2U}
\end{figure}

The two resistors together are a voltage divider. We recognize this from our previous labs. So we expect the voltage drop across each resistor to sum to the voltage given by the power supply

\begin{equation*}
	\Delta V_{ps}=\Delta V_{R}+\Delta V_{s}
\end{equation*}

\noindent and we know the current will be the same in the entire circuit. We can use Ohm's law to find the current.

\begin{equation*}
	\Delta V_{ps}=IR_{total}
\end{equation*}

\noindent so that 

\begin{eqnarray*}
	I &=&\frac{\Delta V_{ps}}{R_{total}} \\
	  &=&\frac{\Delta V_{ps}}{R+R_{s}}
\end{eqnarray*}

Now we can find the voltage drop across just $R_{s}$

\begin{eqnarray*}
	\Delta V_{s} &=&IR_{s} \\
                 &=&\left( \frac{\Delta V_{ps}}{R+R_{s}}\right) R_{s}
\end{eqnarray*}

\noindent and let's put in numbers

\begin{eqnarray*}
	\Delta V_{s} &=&\left(\frac{2\unit{V}}{5000\unit{\Omega}+10\unit{\Omega}}\right) \left( 10\unit{\Omega}\right) \\
    &=&3.\,\allowbreak 992\times 10^{-3}\unit{V} \\
    &=&3.\,\allowbreak 992\unit{mV}
\end{eqnarray*}

\noindent Remember that for our simple voltmeter, 

\begin{equation*}
	\Delta V_{\min }=\frac{5\unit{V}}{1024}=4.\,\allowbreak 88\unit{mV}
\end{equation*}

\noindent and this is larger than $\Delta V_{s}$ so once we use our Arduino analog to digital converter (ADC), $\Delta V_{s}$ will appear to be zero! Our current meter that we built will say our current measurement will be zero even thought there is a current flowing. That is a 100\% error!

We might try to improve things by increasing the power supply voltage. Even
if we increased the voltage from the power supply to, say, $5\unit{V}$ (our
maximum) we would only have 

\begin{eqnarray*}
	\Delta V_{s}   &=&\left(\frac{5\unit{V}}{5000\unit{\Omega}+10\unit{\Omega}}\right) \left( 10\unit{\Omega}\right) \\
    &=&9\,.98\unit{mV}
\end{eqnarray*}

We should compare this value to our ADC minimum detectable value 

\begin{equation*}
	\frac{\Delta V_{s}}{\Delta V_{\min }}=N
\end{equation*}

\noindent the number of ADC units that will be used. We can see that for $R_{s}=10\unit{\Omega}$ 

\begin{equation*}
	N=\frac{9\,.98\unit{mV}}{4.\,\allowbreak 88\unit{mV}}=2
\end{equation*}

\noindent ADC units. With our entire value of $\Delta V_{s}$ split into only two numbers our uncertainty in our $\Delta V_{s}$ would be something like $50\%.$ That won't make a very good current measurement

Suppose instead, we use $R_{s}=170\unit{*\Omega}.$ 

\noindent This is much bigger, so it will affect the voltage measurement of the test resistor a little. But in the end will work better. If we set our power supply back to $\Delta V_{ps}=2\unit{V}$ the $R_{s}=170\unit{\Omega}$ gives. 

\begin{eqnarray*}
	\Delta V_{s} &=&\left( \frac{2\unit{V}}{5000\unit{\Omega}+170\unit{\Omega}}\right) \left( 170\unit{\Omega}\right) \\
                 &=&65.\,\allowbreak 764\unit{mV}
\end{eqnarray*}

\noindent This would give 

\begin{equation*}
	N=\frac{65.\,\allowbreak 764\unit{mV}}{4.\,\allowbreak 88\unit{mV}}=13.\,\allowbreak 476
\end{equation*}

\noindent or about $13$ ADC units spread across our $65.\,\allowbreak 764\unit{mV}.$ Then each of our ADC units would be worth 

\begin{equation*}
	\delta \Delta V_{s}=\frac{65.\,\allowbreak 764\unit{mV}}{13}=5.\,\allowbreak1\unit{mV}
\end{equation*}

This is very near the $\Delta V_{\min }=$ $4.88\,\allowbreak \unit{mV}$ value, but a little bit higher. If $\delta \Delta V_{s}$ from our calculation is larger than $\delta V_{\min ,}$ then we have to use the larger value as our uncertainty in $\Delta V_{s}.$ So we would say $\delta \Delta V_{s}=5.\,\allowbreak 1\unit{mV}.$ But still this is not a terrible
error. 
\begin{equation*}
	100\times \frac{5.\,\allowbreak 058\,8\unit{mV}}{65.\,\allowbreak 764\unit{mV}}=7.\,\allowbreak 7\
\end{equation*}

This is much better than $50\%$ or $100\%$ error. You might guess that we
can do a little better by trying other resistance values. And you would be
right. But if you only need an $8\%$ error, this value would be fine.

Our stand-alone meters have lots of shunt resistors inside of them. You are
changing shunt resistors when you change the dial setting, trying to balance these errors. By changing shunt resistors in our circuit we are doing the same thing as turning the dial on the current settings of a multimeter.

\subsection{Finding Uncertainty in a calculated value}

In the last section I\ gave errors in $\Delta V_{s},$ but didn't finish the
error in the current, $I.$ Of course, since we had to calculate the current, we also need to find the uncertainty in our current using error propagation. Fortunately we ``remember'' how to do this from PH150. We use our basic form for standard error propagation. If we have a function $f\left( x,y,z\right) $ then the uncertainty in $f$ would be 

\begin{equation*}
	\delta f=\sqrt{\left( \left( \frac{\partial f}{\partial x}\right)\left(\delta x\right) \right) ^{2}+\left( \left( \frac{\partial f}{\partial y}\right) \left( \delta y\right) \right) ^{2}+\left( \left( \frac{\partial f}{\partial z}\right) \left( \delta z\right) \right) ^{2}}
\end{equation*}

In our current case, our function $f$ is the current $I$ and it is a function of $\Delta V_{s}$ and $R_{s}$ 

\begin{equation*}
	f=I=\frac{\Delta V_{s}}{R_{s}}
\end{equation*}

\noindent so we will have an uncertainty like this

\begin{equation*}
	\delta I=\sqrt{\left( \left( \frac{\partial I}{\partial \Delta V_{s}}\right) \left( \delta \Delta V_{s}\right) \right) ^{2}+\left( \left( \frac{\partial I}{\partial R_{s}}\right) \left( \delta R_{s}\right) \right) ^{2}}
\end{equation*}

\noindent and we can find the partial derivatives

\begin{equation*}
	\frac{\partial I}{\partial \Delta V_{s}}=\frac{1}{R_{s}}
\end{equation*}

\begin{equation*}
	\frac{\partial I}{\partial R_{s}}=-\frac{\Delta V_{s}}{R_{s}^{2}}
\end{equation*}

\noindent so we have 

\begin{equation*}
	\delta I=\sqrt{\left( \left( \frac{1}{R_{s}}\right) \left( \delta \Delta V_{s}\right) \right) ^{2}+\left( \left( -\frac{\Delta V_{s}}{R_{s}^{2}}\right) \left( \delta R_{s}\right) \right) ^{2}}
\end{equation*}

Let's try this for our example in the last section. We have $\Delta V_{s}=9\,.98\unit{mV}$ and $R_{s}=170\unit{\Omega}.$ We know that $\delta \Delta V_{s}=$ $5.\,\allowbreak 058\,8\unit{mV}$ and our resistors are only good to $1\%$ so that would be $\delta R_{S}=1.7\unit{\Omega}$

\begin{eqnarray*}
	\delta I &=&\sqrt{\left( \left( \frac{1}{170\unit{\Omega}}\right) \left( 5.\,\allowbreak 058\,8\unit{mV}\right) \right) ^{2}+\left(
	\left( -\frac{65.\,\allowbreak 764\unit{mV}}{\left( 170\unit{\Omega}\right) ^{2}}\right) \left( 1.7\unit{\Omega}\right) \right) ^{2}} \\
            &=&3.\,\allowbreak 000\,8\times 10^{-5}\unit{A}
\end{eqnarray*}

This looks small. Is it a good uncertainty? We can't tell until we compare
it to our expected current. We expect for our example 

\begin{eqnarray*}
	I &=&\frac{\Delta V_{ps}}{R} \\
	  &=&\frac{2\unit{V}}{5000\unit{\Omega}} \\
      &=&\allowbreak 0.000\,4\unit{A} \\
      &=&4\times 10^{4}\unit{A}
\end{eqnarray*}

\noindent so the fractional uncertainty in the current will be 

\begin{equation*}
	100\frac{3.\,\allowbreak 000\,8\times 10^{-5}\unit{A}}{4\times 10^{4}\unit{A}}=7.\,\allowbreak 502\
\end{equation*}%


This still isn't great, it's about what we got for the error in $\delta
\Delta V_{s}$ (but it's better than $50\%$). We might be able to do better.
But if $8\%$ is OK for our application, then we stop here!

Let's try to figure out what the biggest contributor to our uncertainty
might be. To do this we look at the terms in our uncertainty calculation
separately

\begin{eqnarray*}
	\left( \left( \frac{1}{R_{s}}\right) \left( \delta \Delta V_{s}\right)\right) ^{2} &=&\left( \left( \frac{1}{170\unit{\Omega}}\right) \left( 5.\,\allowbreak 058\,8\unit{mV}\right) \right) ^{2}=\allowbreak 8.\,\allowbreak 855\,2\times 10^{-10}\unit{A}^{2} \\
	\left( \left( -\frac{\Delta V_{s}}{R_{s}^{2}}\right) \left( \delta R_{s}\right) \right) ^{2} &=&\left( \left( -\frac{65.\,\allowbreak 764\unit{mV}}{\left( 170\unit{\Omega}\right) ^{2}}\right) \left( 1.7\unit{\Omega}.\right) \right) ^{2}=\allowbreak 1.\,\allowbreak 496\,5\times 10^{-11}%\unit{A}^{2}
\end{eqnarray*}

The first term is about sixty times the second. So to make an improvement we would want to first concentrate on the first term. We could change our $\delta \Delta V_{s}$ or change our $R_{s}$ value. Changing $\Delta V_{s}$ is harder than changing $R_{s}.$ Maybe we could even make $R_{s}$ a little
bigger to improve our current measurement. \emph{Notice that this was not
the obvious solution!} At first it seemed that smaller $R_{s}$ values would
give better uncertainties. But after doing the uncertainty calculations, we
find that there is an optimal range for $R_{s}.$ Big $R_{s}$ is still bad,
but very small $R_{s}$ is also bad. You have to do the math to find this out.

\subsubsection{Iterate to find an optimal value}

Since there is an $R_{s}$ in the bottom of both terms in our current uncertainty, let's try changing the $R_{s}$ value and see if the uncertainty gets better. We have to start all the way back at the top with $\Delta V_{s}. $ We will have to go through all our calculations again! A
spreadsheet or symbolic math processor might be a good way to go so you aren't putting the same things in your calculator over and over.

We start by finding the current in the circuit

\begin{equation*}
	I=\left( \frac{\Delta V_{ps}}{R+R_{s}}\right)
\end{equation*}

\noindent Then an estimate for $\Delta V_{s}$ across the shunt resistor would be

\begin{eqnarray*}
	\Delta V_{s} &=&IR_{s} \\ &=&\left( \frac{\Delta V_{ps}}{R+R_{s}}\right) R_{s}
\end{eqnarray*}

\noindent and then the number of ADC units we used will be 

\begin{equation*}
	N_{ADC}=\frac{\Delta V_{s}}{\Delta V_{\min }}
\end{equation*}

\noindent rounded to the smallest integer, which gives a new estimate of our uncertainty in $\Delta V_{s}$ 

\begin{equation*}
	\delta \Delta V_{s}=\frac{\Delta V_{s}}{N_{ADC}}
\end{equation*}

\noindent and now we need can find the uncertainty in $I$

\begin{equation*}
	\delta I=\sqrt{\left( \left( \frac{1}{R_{s}}\right) \left( \delta \Delta V_{s}\right) \right) ^{2}+\left( \left( -\frac{\Delta V_{s}}{R_{s}^{2}} \right) \left( \delta R_{s}\right) \right) ^{2}}
\end{equation*}

\noindent and its fractional uncertainty

\begin{equation*}
	f_{I}=\frac{dI}{I}
\end{equation*}

As you can see, it is probably best to put all this in a symbolic package
(like Mathmatica or Maple, or Sage, or whatever your favorite symbolic math
processor might be). That way, you can change values of, say, $R_{s}$ and $\Delta V_{s}$ without redoing everything. At least consider using a spreadsheet program or even in Python!.

Let's try this once more with $R_{s}=500\unit{\Omega}$ just to see what would happen.

\begin{eqnarray*}
	I &=&\left( \frac{2\unit{V}}{5000\unit{\Omega}+500\unit{\Omega}}\right) \\
      &=&3.\,\allowbreak 636\,4\times 10^{-4}\unit{A}
\end{eqnarray*}

\noindent so

\begin{eqnarray*}
	\Delta V_{s} &=&IR_{s} \\
                 &=&\left( \frac{2\unit{V}}{5000\unit{\Omega}+500\unit{\Omega}}\right) \left( 500\unit{\Omega}\right) \\
                 &=&0.181\,82\unit{V}
\end{eqnarray*}

\noindent and then the number of ADC units we used will be 

\begin{equation*}
     N_{ADC}=\frac{0.181\,82\unit{V}}{4.\,\allowbreak 88\unit{mV}}=37.\,\allowbreak 258
\end{equation*}

\noindent which gives a new estimate of our uncertainty in $\Delta V_{s}$ 

\begin{equation*}
    \delta \Delta V_{s}=\frac{0.181\,82\unit{V}}{37}=4.\,\allowbreak914\,1\times 10^{-3}\unit{V}
\end{equation*}

\noindent and now we need can find the uncertainty in $I.$ We will need $\delta
R_{s}=500\unit{\Omega}\times 0.01=\allowbreak 5.0\unit{\Omega}$

\begin{equation*}
	\delta I=\sqrt{\left( \left( \frac{1}{500\unit{\Omega}}\right) \left( \delta \Delta V_{s}\right) \right) ^{2}+\left( \left( -\frac{\Delta V_{s}}{\left( 500\unit{\Omega}\right) ^{2}}\right) \left( \delta R_{s}\right) \right) ^{2}}
\end{equation*}

\begin{eqnarray*}
	\delta I &=&\sqrt{\left( \left( \frac{1}{500\unit{\Omega}}\right) \left( 4.\,\allowbreak 914\,1\times 10^{-3}\unit{V}\right) \right)^{2}+\left( \left( -\frac{0.181\,82\unit{V}}{\left( 500\unit{\Omega}\right) ^{2}}\right) \left( 5.0\unit{\Omega}\right) \right) ^{2}} \\
            &=&1.\,\allowbreak 047\,9\times 10^{-5}\unit{A}
\end{eqnarray*}

\noindent and its fractional uncertainty

\begin{equation*}
	f_{I}=100\times \frac{1.\,\allowbreak 047\,9\times 10^{-5}\unit{A}}{3.\,\allowbreak 636\,4\times 10^{-4}\unit{A}}=2.\,\allowbreak 881\,7\
\end{equation*}


This was a bit of an improvement! We could continue to iterate. I had my
computer do this for $R=5000\unit{\Omega}$ and $\Delta V_{ps}=2\unit{V}$ I asked it to plot $f_{I}$ as a function of $R_{s}$ 

\begin{figure}[h!]
	\centering
	\includegraphics[width=4.5in,height=1.5in]{PH4CAV2V}
\end{figure}

Notice that after about $500\unit{\Omega}$ we are not going to get much of an improvement. So our choice of $R_{s}=500\unit{\Omega}$ seems good for this situation. Once you have a symbolic package or spreadsheet version of this calculation, picking different shunt resistors becomes fairly easy.

We should check, though. What did our $500\unit{\Omega}$ resistor do to our $\Delta V$ measurement? We found our current in the circuit to be 

\begin{equation*}
	I=3.\,\allowbreak 636\,4\times 10^{-4}\unit{A}
\end{equation*}

\noindent and our test resistor is $5000\unit{\Omega}$ so 

\begin{equation*}
	\Delta V_{test}=\left( 3.\,\allowbreak 636\,4\times 10^{-4}\unit{A}\right)\left( 5000\unit{\Omega}\right) =1.\,\allowbreak 818\,2\unit{V}
\end{equation*}

\noindent We know the power supply was providing $\Delta V_{ps}=2\unit{V}$. So we have introduced an error. We can find the percent difference 

\begin{equation*}
	\left( 100\right) \frac{1.\,\allowbreak 818\,2\unit{V}-2\unit{V}}{1.\,\allowbreak 818\,2\unit{V}}=-9.\,\allowbreak 998\,9\
\end{equation*}

\noindent which means the $\Delta V$ measurement will be $10\%$ low due to our inserting the shunt resistance. If we can live with a $10\%$ error, then we are fine. If not, it is back to iteration to find a better shunt resistance.

Of course, so far we have just found uncertainty in $\Delta V$ and $I$.
These are the uncertainties in our measuring devices that we built. Since
you are the manufacturer of these devices, you have had to calculate what
their uncertainties will be. When we design our own measuring devices, we
always have to do this. Of course you could have built the devices and then
watched the output to see where the digits fluctuate like we did with our
stand-alone multimeters. But the risk is that it might take a long time to
find a value for each part of our device that works together with the other
parts, and in the mean time we might burn up our equipment if we don't plan
for what we want first. You can check your uncertainty calculations by
looking at the fluctuation of the digits to see if we are right (or if some
other uncertainty factor has crept in that we haven't handled yet).

When we started this lab we said we were testing Ohm's law, and we wanted to find $R_{test}$ uncertainty $\delta R_{test}$ to see if Ohm's law really
works. In finding the uncertainty in our measuring devices we haven't found $\delta R_{test}.$ We will review a different way to do that analysis.

%%%%%%%%%%%%%%%%%%%%%%%%%%%%%%%%%%%%%%%%%%%%%%%%%%%%%%%%%%%%%%%%%%%%%%%
% Insert a linear fit module here. Your choices as of this writing are
%   a linear fit in excel 
%\input{Chapters/Module_Linear_fit_in_Excel.tex}
%   or a linear fit in python
\subsection{Using statistics to calculate uncertainty}

By now in a experimental design, I am usually at my tolerance limit for calculating uncertainties. You might ask, can't we get our powerful computers to help us out a little bit with finding the uncertainty? After all, we went to all the trouble to get the data on the computer. The answer is, yes!

Let's suppose we have done our experiment and we have some data that look like this: 

\begin{figure}[h!]
	\centering
	\includegraphics[width=3.563in,height=2.7345in]{linear_data}
\end{figure}

This looks pretty good. It seems to be sort of linear. We might guess from this that Ohm's law is begin obeyed. But we want $R_{test}$ and $R_{test}$ is the slope of this line. We could calculate $R_{test}$ from each pair of $\Delta V$ and $I$ points and find it's uncertainty using standard error propagation. But it seems that it would be better to take all the points into our analysis to find $R_{test}.$ More data should give us a better estimate for $R_{test}.$ Back in PH150 you may have done such a thing using a \emph{curve fit}. In the next figure, a curve fit is shown for the data from the last figure. 

\begin{figure}[h!]
	\centering
	\includegraphics[width=3.563in,height=2.7345in]{linear_data_with_fit}
\end{figure}

Notice that I performed this curve fit in python, but if you are a not a great Python programer you could do this in Excel. You could also do this in LoggerPro or many other data analysis programs. If you need help finding a way to perform a curve fit, talk to your instructor. Here is an example code for doing the curve fit in python.

\vspace{0.24in}
\href{https://raw.githubusercontent.com/rtlines/IntermediateLabPH250/main/Code/linear_fit.py}{Download here}

\lstinputlisting[language=Arduino]{Code/linear_fit.py}

The scipy linregress() function uses the equation for a line 

\begin{equation*}
	y = result.slope*x + result.intercept
\end{equation*}

\noindent Our code gives the result

\begin {verbatim}
slope is   0.7737292152727273 +-  0.1612215125332822
intercept is   1.2996514309090914 +-  0.9537993308988922
program ended successfully
\end{verbatim}

\noindent which tells us that we have a slope of about $0.774.$ Since our graph
has $\Delta V$ on the vertical axis and $I$ on the horizontal axis we
recognize 
\begin{eqnarray*}
	\Delta V &=&R_{test}I+0 \\
           y &=&mx+b
\end{eqnarray*}

that $R_{test}$ must be the slope. We can see that the resistance is about a tiny $0.774\unit{\Omega}$ because that is the slope of our fit line. But we know we need an uncertainty along with this nominal value. And low and behold! the scipy lineregress() function has already given us the data we need to find the errors on the slope and intercept as well! This was pretty easy!

Notice that in this analysis technique, we can often afford some error in our 
$\Delta V$ and $I$ values. So maybe a $9\%$ error (like we found in one of
our instrument designs) is not so bad. We may not have to work too hard to
get a wonderful choice of shunt resistor if we are going to use many data
points and the power of statistics to analyze the data in the end!

%%%%%%%%%%%%%%%%%%%%%%%%%%%%%%%%%%%%%%%%%%%%%%%%%%%%%%%%%%%%%%%%%%%%%%%

\subsection{Philosophical warning}

There was a lot to this reading. We talked about designing and building an
instrument. We used some physics, Ohm's law, in our design process. Then we
tested a physical model, Ohm's law, with our instrument. All these were good things, but maybe you wondered along the way if it is acceptable to test Ohm's law with an instrument that depends on Ohm's law. And the answer is a big NO!

This lab is practice, but it is imperfect practice. We do know that Ohm's
law works, so we are going to use it in designing instruments. But you
really need a different instrument, one that does not depend on Ohm's law,
to test Ohm's law in a credible way. In next week's lab, we will test a
different physical model with the same basic instrument that we build today. The instrument will depend on Ohm's law, but the new physical model must not if the experiment is to be valid.

%%%%%%%%%%%%%%%%%%%%%%%%%%%%%%%%%%%%%%%%%%%%%%%%%%%%%%%%%%%%%%%%%%%%%%%
% Normal semesters the proposal section goes here
%%%%%%%%%%%%%%%%%%%%%%%%%%%%%%%%%%%%%%%%%%%%%%%%%%%%%%%%%%%%%%%%%%%%%%%

\section{Lab Assignment}

\begin{enumerate}
\item Build the instrument

\begin{enumerate}
\item Choose a test resistor in the $1\unit{k\Omega}$ to $10\unit{k\Omega}$ range and a shunt resistor. You will have to check your values using the math we discussed above to make sure they will work. If your first shunt resistor choice works, use it. If not, iterate until you have a shunt resistance that will work.

	\item Modify your voltmeter sketch to measure both the voltage and the current. (Check the voltage, currents, and their uncertainties with the serial monitor to make sure things seem good).

	\item Build your voltmeter and ammeter so your Arduino is taking $\left(I,\Delta V\right) $ pares and reporting them. Reporting to the serial monitor is fine for a start. 
	\begin{figure}[h!]
		\centering
		\includegraphics[width=1.8507in,height=0.9677in]{PH4CAW36}
	\end{figure}
	\noindent if you based your sketch on the simple voltmeter, make sure you don't use voltages outside the $0\unit{V}$ to $+5\unit{V}$ range! Include expected uncertainties for your $\Delta V$ and $I$ measurements.

	\item Check your lab group's instruments to see if they work, and have your lab group members check yours.
\end{enumerate}

\item Test Ohm's law

\begin{enumerate}
	\item Take 10-15 measurements of $\Delta V$ and $I.$ For each $\Delta V$ measurement change the $\Delta V$ setting on the power supply a small amount (don't go over $5\unit{V}$ if you are using the simple voltmeter!).
	
	\item Plot voltage vs. current and fit a curve to the data.

	\item Determine the resistance from this curve fit and its uncertainty.

	\item Finally, determine if your results support the Ohm model for how potential and current are related?

	\item Compare your data and conclusions to the data and conclusions of your lab group members. Have them look at your results as well.
\end{enumerate}

	\item If you still have time, repeat part 2 for a diode. Do your results support the Ohm model for how potential and current are related?

	\item Could you have your Arduino sketch report the calculated uncertainties for $\Delta V,$ $I,$ and $R?$ If you have time (you probably won't) give this a try.
\end{enumerate}


\vfill

