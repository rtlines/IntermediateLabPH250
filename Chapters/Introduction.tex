%Introduction
\chapter{Introduction}
As a PH250 student, you are probably taking PH220 concurrently with PH250. This lab course is designed to teach electronics and computer skills while your PH220 professor teaches electromagnetic field theory. Once you have a little bit of electric field theory under your belt from PH220, then our experiments designed to test out models of electric charge and electromagnetic fields begin in earnest. While we are waiting we will spend some time learning about how to control experiments with a computer, and how to import data from an experiment to a computer.

You should read the material for each lab before the lab begins. There will sometimes be practice problems to do to make sure you will be effective in lab. By preparing before lab you will have the full 2 hours and 45 minutes to make sure you can finish the lab work. Some labs may go fast, but most take the entire lab period. I also suggest you practice your computer and electronics skills a little. Build some blinking lights for your apartment, or measure how loud your roommates are, or something. The Arduino can be the data collection and control part of thousands fun projects.

This class is sometimes frustrating. But it is also a lot of fun. You will be introduced to computer instrumentation and will be able to perform an experiment that you and your lab group design. The student designed experiments are only limited by your imagination and our ability to find equipment. If you have concerns during the semester, don't hesitate to find your instructor or TA and ask. 