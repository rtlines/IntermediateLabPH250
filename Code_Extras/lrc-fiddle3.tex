
\documentclass{sebase}
%%%%%%%%%%%%%%%%%%%%%%%%%%%%%%%%%%%%%%%%%%%%%%%%%%%%%%%%%%%%%%%%%%%%%%%%%%%%%%%%%%%%%%%%%%%%%%%%%%%%%%%%%%%%%%%%%%%%%%%%%%%%%%%%%%%%%%%%%%%%%%%%%%%%%%%%%%%%%%%%%%%%%%%%%%%%%%%%%%%%%%%%%%%%%%%%%%%%%%%%%%%%%%%%%%%%%%%%%%%%%%%%%%%%%%%%%%%%%%%%%%%%%%%%%%%%
\usepackage{LECTURENOTES}

%TCIDATA{OutputFilter=LATEX.DLL}
%TCIDATA{Version=5.50.0.2960}
%TCIDATA{<META NAME="SaveForMode" CONTENT="1">}
%TCIDATA{BibliographyScheme=Manual}
%TCIDATA{Created=Monday, June 14, 2010 14:35:44}
%TCIDATA{LastRevised=Wednesday, June 07, 2023 14:18:41}
%TCIDATA{<META NAME="GraphicsSave" CONTENT="32">}
%TCIDATA{<META NAME="DocumentShell" CONTENT="Style Editor\TLLectureNotes">}
%TCIDATA{Language=American English}
%TCIDATA{CSTFile=AG.cst}

\input{tcilatex}
\begin{document}


Today I am supposed to verify our physics model for magnetic induction. We
are supposed to make an LRC circuit with a known capacitor value and a coil
for the inductor, but we don't know the inductance of the coil. We can
calcualte the coil inductance by measuring the coil number of turns and
dimensions. Then we can hook the LRC circuit to a signal generator and an
oscilloscope and look for it's resonance frequency. Knowing the resonance
frequency and the capacitance we can calculate the inductance and compare
this to our inductance calculation. If they match, we have not falsified the
inductance model (so we contintue to have confidence in the model).

Another group told me that the inductance of these coils is 

$L=\frac{2.\,\allowbreak 305\,7\times 10^{-3}}{\unit{A}^{2}}\frac{\unit{m}%
^{2}}{\unit{s}^{2}}\unit{kg}$

and my capacitor today has

$C=26.4\times 10^{-9}\unit{F}$

according to my multimeter. i know it is dangerous to trust the other
group's value, but let's see what resonant frequency it would give. 

$f=\frac{1}{2\pi \sqrt{\left( L\right) \left( C\right) }}$

$f=\frac{1}{2\pi \sqrt{\left( \frac{2.\,\allowbreak 305\,7\times 10^{-3}}{%
\unit{A}^{2}}\frac{\unit{m}^{2}}{\unit{s}^{2}}\unit{kg}\right) \left(
26.4\times 10^{-9}\unit{F}\right) }}\bigskip $ : $\frac{20399.}{\unit{s}}$

I measusred my resonant frequency to be $16.09\unit{kHz}$

So we are off by $\frac{20399.}{\unit{s}}-16.09\unit{kHz}=\allowbreak \frac{%
4309.0}{\unit{s}}$ That seems like a lot

\bigskip Let's check the inductance estimate. First, let me put our
inductance equation into terms I\ can measure

$L=\mu _{o}n^{2}V=\mu _{o}A\ell =\mu _{o}n^{2}\pi r^{2}\ell =\mu _{o}\frac{%
N^{2}}{\ell ^{2}}\pi r^{2}\ell =\mu _{o}\frac{N^{2}}{\ell }\pi r^{2}=\mu _{o}%
\frac{N^{2}}{\ell }\pi \frac{D^{2}}{4}$

\bigskip $N=4\times 104=\allowbreak 416.0$

\bigskip I count 103 turns per layer and I\ estimate 4 layers

\bigskip $N=4\times 104=\allowbreak 416.0$

$\mu _{o}=\allowbreak \frac{1.\,\allowbreak 256\,6\times 10^{-6}}{\unit{A}%
^{2}}\frac{\unit{m}}{\unit{s}^{2}}\unit{kg}$

$D=0.045\unit{m}$

$\ell =0.147\unit{m}$

$L=\mu _{o}\frac{N^{2}}{\ell }\pi r^{2}=\left( \allowbreak \frac{%
1.\,\allowbreak 256\,6\times 10^{-6}}{\unit{A}^{2}}\frac{\unit{m}}{\unit{s}%
^{2}}\unit{kg}\right) \left( \frac{\left( \allowbreak 416.0\right) ^{2}}{%
0.147\unit{m}}\right) \left( \pi \right) \left( \frac{\left( 0.045\unit{m}%
\right) ^{2}}{4}\right) =\allowbreak \frac{2.\,\allowbreak 352\,8\times
10^{-3}}{\unit{A}^{2}}\frac{\unit{m}^{2}}{\unit{s}^{2}}\unit{kg}$

Now try this for our resonant frequency 

\bigskip $f=\frac{1}{2\pi \sqrt{\left( \allowbreak \frac{2.\,\allowbreak
352\,8\times 10^{-3}}{\unit{A}^{2}}\frac{\unit{m}^{2}}{\unit{s}^{2}}\unit{kg}%
\right) \left( 26.4\times 10^{-9}\unit{F}\right) }}\bigskip $ : $\frac{20194.%
}{\unit{s}}$ 

This seems like our original estimate from the other group and might be too
far off. We don't know until we do the uncertainies. But we could try a
trick to check our value. We could solve for the inductance knowing the
resonant frequency and see what value of inductance that would give

$f=\frac{1}{2\pi \sqrt{\left( L\right) \left( C\right) }}$

$f^{2}=\frac{1}{4\pi ^{2}\left( L\right) \left( C\right) }$

$L=\frac{1}{4\pi ^{2}\left( f^{2}\right) \left( C\right) }$

\bigskip $L=\frac{1}{4\pi ^{2}\left( \left( 16.09\unit{kHz}\right)
^{2}\right) \left( 26.4\times 10^{-9}\unit{F}\right) }=\allowbreak \frac{%
3.\,\allowbreak 706\,2\times 10^{-3}}{\unit{A}^{2}}\frac{\unit{m}^{2}}{\unit{%
s}^{2}}\unit{kg}$

This is higher than the inductance that the resonant frequency gives. Maybe
my estimate of the number of layers in the coil is bad. Let's try changing
the number of layers. We really can't tell what the number of layers should
be just from looking at the coil and the professor won't let me take it
apart. 

Suppose there are only three layers

$N=3\times 104=\allowbreak 312$

$L=\mu _{o}\frac{N^{2}}{\ell }\pi r^{2}=\mu _{o}\frac{N^{2}}{\ell }\pi \frac{%
D^{2}}{4}=\left( \allowbreak \frac{1.\,\allowbreak 256\,6\times 10^{-6}}{%
\unit{A}^{2}}\frac{\unit{m}}{\unit{s}^{2}}\unit{kg}\right) \left( \frac{%
\left( 312\right) ^{2}}{0.147\unit{m}}\right) \left( \pi \right) \left( 
\frac{\left( 0.045\unit{m}\right) ^{2}}{4}\right) =\allowbreak \frac{%
1.\,\allowbreak 323\,4\times 10^{-3}}{\unit{A}^{2}}\frac{\unit{m}^{2}}{\unit{%
s}^{2}}\unit{kg}$

and then our prediction for the resonant frequency would be

$f=\frac{1}{2\pi \sqrt{\left( \allowbreak \frac{1.\,\allowbreak 323\,4\times
10^{-3}}{\unit{A}^{2}}\frac{\unit{m}^{2}}{\unit{s}^{2}}\unit{kg}\right)
\left( 26.4\times 10^{-9}\unit{F}\right) }}=\allowbreak \frac{26926.}{\unit{s%
}}$

That got worse! Suppose there are five layers

$N=5\times 104=\allowbreak 520$

$L=\mu _{o}\frac{N^{2}}{\ell }\pi r^{2}=\left( \allowbreak \frac{%
1.\,\allowbreak 256\,6\times 10^{-6}}{\unit{A}^{2}}\frac{\unit{m}}{\unit{s}%
^{2}}\unit{kg}\right) \left( \frac{\left( \allowbreak 520\right) ^{2}}{0.147%
\unit{m}}\right) \left( \pi \right) \left( \frac{\left( 0.045\unit{m}\right)
^{2}}{4}\right) =\allowbreak \frac{3.\,\allowbreak 676\,2\times 10^{-3}}{%
\unit{A}^{2}}\frac{\unit{m}^{2}}{\unit{s}^{2}}\unit{kg}$

$f=\frac{1}{2\pi \sqrt{\left( \frac{3.\,\allowbreak 676\,2\times 10^{-3}}{%
\unit{A}^{2}}\frac{\unit{m}^{2}}{\unit{s}^{2}}\unit{kg}\right) \left(
26.4\times 10^{-9}\unit{F}\right) }}=\allowbreak \frac{16155.}{\unit{s}}$

That seems much better!

We just need to do the uncertainty calculations to see if we are OK with our
model and the measurement.

\bigskip Let's start with 

\bigskip $L=\mu _{o}\frac{\left( layers\times count\right) ^{2}}{\ell }\pi 
\frac{D^{2}}{4}$

The $count$ is a pure integer, so ideally there is no uncertainty. But we
don't know how many layers there are. We ended up picking $5$ but it could
be $\pm 1$ layer at least. $\mu _{o}$ is 

$\mu _{o}=\left( 1.\,\allowbreak 256\,6\times 10^{-6}\pm 1.9\times
10^{-16}\right) \allowbreak \frac{\unit{N}}{\unit{A}^{2}}$

Our length is 

$\ell =\left( 0.147\pm 0.005\right) \unit{m}$

and $\pi $ we can take to as many digits as we need. The diameter $D$ is 

$D=\left( 0.045\pm 0.05\right) \unit{m}$ 

where I was just not sure where to stop with the windings in layers and the
plastic spool end hiding the layers.

Then our uncertainty in $L$ would be

$\delta L=\sqrt{\left( \frac{\partial L}{\partial \mu _{o}}\delta \mu
_{o}\right) ^{2}+\left( \frac{\partial L}{\partial layers}\delta
layers\right) ^{2}+\left( \frac{\partial L}{\partial count}\delta
count\right) ^{2}+\left( \frac{\partial L}{\partial \ell }\delta \ell
\right) ^{2}+\left( \frac{\partial L}{\partial D}\delta D\right) ^{2}}$

$\delta L=\sqrt{\left( \frac{\left( layers\times count\right) ^{2}}{\ell }%
\pi \frac{D^{2}}{4}\delta \mu _{o}\right) ^{2}+\left( \mu _{o}\frac{2\left(
layers\times count\right) }{\ell }\pi \frac{D^{2}}{4}\delta layers\right)
^{2}+\left( \mu _{o}\frac{2\left( layers\times count\right) }{\ell }\pi 
\frac{D^{2}}{4}\delta count\right) ^{2}+\left( \mu _{o}\frac{\left( \left(
layers\times count\right) \right) ^{2}}{\ell 2}\pi \frac{D^{2}}{4}\delta
\ell \right) ^{2}+\left( \mu _{o}\frac{\left( l\times c\right) ^{2}}{\ell }%
\pi \frac{D}{2}\delta D\right) ^{2}}$

This looks messy enought that I want to do each term separately then combine
them in the end

$\left( \frac{\left( layers\times count\right) ^{2}}{\ell }\pi \frac{D^{2}}{4%
}\delta \mu _{o}\right) ^{2}=\left( \frac{\left( 5\times 104\right) ^{2}}{%
0.147\unit{m}}\pi \frac{\left( 0.45\unit{m}\right) ^{2}}{4}\delta \mu
_{o}\right) ^{2}=\left( \left( \frac{\left( \allowbreak 520\right) ^{2}}{%
0.147\unit{m}}\right) \left( \pi \right) \left( \frac{\left( 0.045\unit{m}%
\right) ^{2}}{4}\right) \left( 1.9\times 10^{-16}\allowbreak \frac{\unit{N}}{%
\unit{A}^{2}}\right) \right) ^{2}=\allowbreak \frac{3.\,\allowbreak
089\,7\times 10^{-25}}{\unit{A}^{4}}\frac{\unit{m}^{4}}{\unit{s}^{4}}\unit{kg%
}^{2}$

$\left( \mu _{o}\frac{2\left( layers\times count\right) }{\ell }\pi \frac{%
D^{2}}{4}\delta layers\right) ^{2}=\left( \left( \allowbreak \frac{%
1.\,\allowbreak 256\,6\times 10^{-6}}{\unit{A}^{2}}\frac{\unit{m}}{\unit{s}%
^{2}}\unit{kg}\right) \frac{2\left( 5\times 104\right) }{0.147\unit{m}}\pi 
\frac{\left( 0.045\unit{m}\right) ^{2}}{4}\left( 1\right) \right)
^{2}=\allowbreak \frac{1.\,\allowbreak 999\,2\times 10^{-10}}{\unit{A}^{4}}%
\frac{\unit{m}^{4}}{\unit{s}^{4}}\unit{kg}^{2}$

$\left( \mu _{o}\frac{2\left( layers\times count\right) }{\ell }\pi \frac{%
D^{2}}{4}\delta count\right) ^{2}=\left( \left( \allowbreak \frac{%
1.\,\allowbreak 256\,6\times 10^{-6}}{\unit{A}^{2}}\frac{\unit{m}}{\unit{s}%
^{2}}\unit{kg}\right) \frac{2\left( 5\times 104\right) }{0.147\unit{m}}\pi 
\frac{\left( 0.045\unit{m}\right) ^{2}}{4}\left( 0\right) \right)
^{2}=\allowbreak 0.0$

$\left( \mu _{o}\frac{\left( \left( layers\times count\right) \right) ^{2}}{%
\ell ^{2}}\pi \frac{D^{2}}{4}\delta \ell \right) ^{2}=\left( \left(
\allowbreak \frac{1.\,\allowbreak 256\,6\times 10^{-6}}{\unit{A}^{2}}\frac{%
\unit{m}}{\unit{s}^{2}}\unit{kg}\right) \frac{\left( 5\times 104\right) ^{2}%
}{\left( 0.147\unit{m}\right) ^{2}}\pi \frac{\left( 0.045\unit{m}\right) ^{2}%
}{4}\left( 0.005\unit{m}\right) \right) ^{2}=\allowbreak \frac{%
1.\,\allowbreak 563\,5\times 10^{-8}}{\unit{A}^{4}}\frac{\unit{m}^{4}}{\unit{%
s}^{4}}\unit{kg}^{2}$

$\left( \mu _{o}\frac{\left( l\times c\right) ^{2}}{\ell }\pi \frac{D}{2}%
\delta D\right) =\left( \left( \allowbreak \frac{1.\,\allowbreak
256\,6\times 10^{-6}}{\unit{A}^{2}}\frac{\unit{m}}{\unit{s}^{2}}\unit{kg}%
\right) \frac{\left( 5\times 104\right) ^{2}}{0.147\unit{m}}\pi \frac{\left(
0.045\unit{m}\right) }{2}\left( 0.01\unit{m}\right) \right) ^{2}\allowbreak
=\allowbreak \frac{2.\,\allowbreak 669\,5\times 10^{-6}}{\unit{A}^{4}}\frac{%
\unit{m}^{4}}{\unit{s}^{4}}\unit{kg}^{2}$

We can see that getting a better estimate for the diameter of the coil would
improve this quite a bit. But there is a fundamental limit in that the coils
are really not all at one diameter. We should probably integrate over a
range of diameters. But let's not for today. Let's see if we can make this
work to a reasonable uncertainty with our simple model that all the coils
are at one diameter. 

Then 

$\delta L=\sqrt{\allowbreak \allowbreak \frac{3.\,\allowbreak 089\,7\times
10^{-25}}{\unit{A}^{4}}\frac{\unit{m}^{4}}{\unit{s}^{4}}\unit{kg}^{2}+\frac{%
1.\,\allowbreak 999\,2\times 10^{-10}}{\unit{A}^{4}}\frac{\unit{m}^{4}}{%
\unit{s}^{4}}\unit{kg}^{2}+\allowbreak 0.0+\frac{1.\,\allowbreak
563\,5\times 10^{-8}}{\unit{A}^{4}}\frac{\unit{m}^{4}}{\unit{s}^{4}}\unit{kg}%
^{2}+\frac{2.\,\allowbreak 669\,5\times 10^{-6}}{\unit{A}^{4}}\frac{\unit{m}%
^{4}}{\unit{s}^{4}}\unit{kg}^{2}}\allowbreak $ : $\frac{1.\,\allowbreak
638\,7\times 10^{-3}}{\unit{A}^{2}}\frac{\unit{m}^{2}}{\unit{s}^{2}}\unit{kg}
$

so our value for $L$ is 

$L=\left( \allowbreak \frac{3.\,\allowbreak 676\,2\times 10^{-3}}{\unit{A}%
^{2}}\frac{\unit{m}^{2}}{\unit{s}^{2}}\unit{kg}\right) \pm (\allowbreak
\allowbreak \allowbreak \allowbreak \frac{1.\,\allowbreak 638\,7\times
10^{-3}}{\unit{A}^{2}}\frac{\unit{m}^{2}}{\unit{s}^{2}}\unit{kg})$ which is
really not a great measurement. Let's calcualte a fractional uncertainty to
see how bad it is.%
\[
\frac{\delta L}{L}=\frac{(\frac{1.\,\allowbreak 638\,7\times 10^{-3}}{\unit{A%
}^{2}}\frac{\unit{m}^{2}}{\unit{s}^{2}}\unit{kg})}{\left( \allowbreak \frac{%
3.\,\allowbreak 676\,2\times 10^{-3}}{\unit{A}^{2}}\frac{\unit{m}^{2}}{\unit{%
s}^{2}}\unit{kg}\right) }=0.445\,76
\]
So we have a 45\% error. Nothing to write home about. But let's add it to
our uncertainty in our frequency to see what we can get.

$f=\frac{1}{2\pi \sqrt{\left( \allowbreak \frac{3.\,\allowbreak 676\,2\times
10^{-3}}{\unit{A}^{2}}\frac{\unit{m}^{2}}{\unit{s}^{2}}\unit{kg}\right)
\left( 26.4\times 10^{-9}\unit{F}\right) }}=\allowbreak \frac{16155.}{\unit{s%
}}$

so $\delta f$ would be

$\delta f=\sqrt{\left( \frac{\partial f}{\partial L}\delta L\right)
^{2}+\left( \frac{\partial f}{\partial C}\delta C\right) }$

We found our inductance to be

$L=\left( \allowbreak \frac{3.\,\allowbreak 676\,2\times 10^{-3}}{\unit{A}%
^{2}}\frac{\unit{m}^{2}}{\unit{s}^{2}}\unit{kg}\right) \pm (\frac{%
1.\,\allowbreak 638\,7\times 10^{-3}}{\unit{A}^{2}}\frac{\unit{m}^{2}}{\unit{%
s}^{2}}\unit{kg})$ 

and our capacitor is a 20\% uncertainty 

$C=26.4\times 10^{-9}\unit{F}\pm $

$\delta C=0.2\times 26.4\times 10^{-9}\unit{F}=\allowbreak 5.\,\allowbreak
28\times 10^{-9}\unit{F}$

So we have 

$\delta f=\sqrt{\left( \frac{\partial f}{\partial L}\delta L\right)
^{2}+\left( \frac{\partial f}{\partial C}\delta C\right) ^{2}}$

$\frac{\partial f}{\partial L}=\frac{1}{2\pi }\frac{C}{2\left( LC\right) ^{%
\frac{3}{2}}}$

$\frac{\partial }{\partial x}\frac{1}{\sqrt{x}}=\allowbreak -\frac{1}{2x^{%
\frac{3}{2}}}$

$\frac{\partial }{\partial L}\frac{1}{2\pi }\frac{1}{\left( LC\right) ^{%
\frac{1}{2}}}=\allowbreak -\frac{1}{4\pi }\frac{C}{\left( CL\right) ^{\frac{3%
}{2}}}$

\bigskip $\delta f=\sqrt{\left( -\frac{1}{4\pi }\frac{C}{\left( CL\right) ^{%
\frac{3}{2}}}\delta L\right) ^{2}+\left( -\frac{1}{4\pi }\frac{L}{\left(
CL\right) ^{\frac{3}{2}}}\delta C\right) ^{2}}$

\bigskip $\left( -\frac{1}{4\pi }\frac{26.4\times 10^{-9}\unit{F}}{\left(
\left( 26.4\times 10^{-9}\unit{F}\right) \left( \allowbreak \frac{%
3.\,\allowbreak 676\,2\times 10^{-3}}{\unit{A}^{2}}\frac{\unit{m}^{2}}{\unit{%
s}^{2}}\unit{kg}\right) \right) ^{\frac{3}{2}}}(\frac{1.\,\allowbreak
638\,7\times 10^{-3}}{\unit{A}^{2}}\frac{\unit{m}^{2}}{\unit{s}^{2}}\unit{kg}%
)\right) ^{2}=\allowbreak \frac{1.\,\allowbreak 296\,5\times 10^{7}}{\unit{s}%
^{2}}$

$\left( \left( -\frac{1}{4\pi }\frac{\left( \frac{3.\,\allowbreak
676\,2\times 10^{-3}}{\unit{A}^{2}}\frac{\unit{m}^{2}}{\unit{s}^{2}}\unit{kg}%
\right) }{\left( \left( 26.4\times 10^{-9}\unit{F}\right) \left( \allowbreak 
\frac{3.\,\allowbreak 676\,2\times 10^{-3}}{\unit{A}^{2}}\frac{\unit{m}^{2}}{%
\unit{s}^{2}}\unit{kg}\right) \right) ^{\frac{3}{2}}}(\allowbreak
5.\,\allowbreak 28\times 10^{-9}\unit{F})\right) \right) ^{2}=\allowbreak 
\frac{2.\,\allowbreak 610\,0\times 10^{6}}{\unit{s}^{2}}$

\bigskip $\delta f=\sqrt{\allowbreak \allowbreak \allowbreak \allowbreak
\allowbreak \frac{1.\,\allowbreak 296\,5\times 10^{7}}{\unit{s}^{2}}%
+\allowbreak \frac{2.\,\allowbreak 610\,0\times 10^{6}}{\unit{s}^{2}}}%
=\allowbreak \frac{3946.\,\allowbreak 5}{\unit{s}}$

So our predicted frequency is 

$f_{predicted}=\frac{1615.\,\allowbreak 5.}{\unit{s}}\pm \allowbreak \frac{%
4800.\,\allowbreak 9}{\unit{s}}$

and our measured frequency is probaby good to 1\% from our signal generator

$f_{measured}=16.09\unit{kHz}$

$16.09\unit{kHz}\times 0.1=\allowbreak 0.1\allowbreak 609\unit{kHz}$

\bigskip $f_{measured}=16.09\unit{kHz}\pm 0.\allowbreak 1\allowbreak 609%
\unit{kHz}$

So here is what we have

Points

$%
\begin{array}{cc}
1 & 16.09 \\ 
2 & 16.15\,\allowbreak 5%
\end{array}%
$

Error Bars

$%
\begin{array}{cc}
1 & 16.09-0.\allowbreak 1\allowbreak 609 \\ 
1 & 16.09+0.\allowbreak 1\allowbreak 609%
\end{array}%
$

$%
\begin{array}{cc}
2 & 16.15\,\allowbreak 5+3.946\allowbreak 5 \\ 
2 & 16.15\,\allowbreak 5-3.946\allowbreak 5%
\end{array}%
$\FRAME{dtbpFX}{2.4232in}{3in}{0pt}{}{}{Plot}{\special{language "Scientific
Word";type "MAPLEPLOT";width 2.4232in;height 3in;depth 0pt;display
"USEDEF";plot_snapshots TRUE;mustRecompute FALSE;lastEngine "MuPAD";xmin
"0";xmax "3";xviewmin "0";xviewmax "3";yviewmin "5";yviewmax
"23";viewset"XY";rangeset"X";plottype 4;plotticks 1;num-x-ticks
1;labeloverrides 3;axesFont "Times New
Roman,12,0000000000,useDefault,normal";numpoints 100;plotstyle
"patchnogrid";axesstyle "normal";axestips FALSE;xis \TEXUX{x};var1name
\TEXUX{$x$};function
\TEXUX{$\MATRIX{2,2}{c}\VR{,,c,,,}{,,c,,,}{,,,,,}\HR{,,}\CELL{1}\CELL{16.09}%
\CELL{2}\CELL{16.15\,\allowbreak 5}$};linecolor "black";linestyle
1;pointplot TRUE;pointstyle "point";linethickness 3;lineAttributes
"Solid";curveColor "[flat::RGB:0000000000]";curveStyle "Point";function
\TEXUX{$\MATRIX{2,2}{c}\VR{,,c,,,}{,,c,,,}{,,,,,}\HR{,,}\CELL{1}%
\CELL{16.09-0.\allowbreak 1\allowbreak 609}\CELL{1}\CELL{16.09+0.\allowbreak
1\allowbreak 609}$};linecolor "blue";linestyle 1;pointstyle
"point";linethickness 3;lineAttributes "Solid";curveColor
"[flat::RGB:0x000000ff]";curveStyle "Line";function
\TEXUX{$\MATRIX{2,2}{c}\VR{,,c,,,}{,,c,,,}{,,,,,}\HR{,,}\CELL{2}\CELL{16.15%
\,\allowbreak 5+3.946\allowbreak 5}\CELL{2}\CELL{16.15\,\allowbreak
5-3.946\allowbreak 5}$};linecolor "blue";linestyle 1;pointstyle
"point";linethickness 3;lineAttributes "Solid";curveColor
"[flat::RGB:0x000000ff]";curveStyle "Line";VCamFile
'RVWGF513.xvz';valid_file "T";tempfilename
'RVWGF507.wmf';tempfile-properties "XPR";}}

This looks pretty good. There were a lot of assumptions, but so far we have
not shown our inductance model to be too far wrong.

OK\ so I forgot that I\ am supposed to compare inductances not frequencies,
so I\ need to calculate the inductance from the measured frquency and it's
uncertainty. Part of this I\ already did.

$L=\frac{1}{4\pi ^{2}\left( f^{2}\right) \left( C\right) }$

\bigskip $L=\frac{1}{4\pi ^{2}\left( \left( 16.09\unit{kHz}\right)
^{2}\right) \left( 26.4\times 10^{-9}\unit{F}\right) }=\allowbreak \frac{%
3.\,\allowbreak 706\,2\times 10^{-3}}{\unit{A}^{2}}\frac{\unit{m}^{2}}{\unit{%
s}^{2}}\unit{kg}$

but I\ need the uncertainty.

$\delta L_{A}=\sqrt{\left( \frac{\partial L_{A}}{\partial f}\delta f\right)
^{2}+\left( \frac{\partial L_{A}}{\partial C}\delta C\right) ^{2}}$

and 

$\frac{\partial L_{A}}{\partial f}=-\frac{1}{4\pi ^{2}\left( f^{3}\right)
\left( C\right) }$

$\frac{\partial L_{A}}{\partial C}=-\frac{1}{4\pi ^{2}\left( f^{2}\right)
\left( C^{2}\right) }$

and our measured frequency is probaby good to 1\% from our signal generator

$f_{measured}=16.09\unit{kHz}$

$16.09\unit{kHz}\times 0.01=\allowbreak \allowbreak 0.01\allowbreak 609\unit{%
kHz}$

\bigskip $f_{measured}=16.09\unit{kHz}\pm 0.\allowbreak 1\allowbreak 609%
\unit{kHz}$

and our capacitor is a 20\% capicitor

$\delta C=0.2\times 26.4\times 10^{-9}\unit{F}=\allowbreak 5.\,\allowbreak
28\times 10^{-9}\unit{F}$

So we get

$\delta L_{A}=\sqrt{\left( -\frac{1}{4\pi ^{2}\left( f^{3}\right) \left(
C\right) }\delta f\right) ^{2}+\left( -\frac{1}{4\pi ^{2}\left( f^{2}\right)
\left( C^{2}\right) }\delta C\right) ^{2}}$

\bigskip 

$\left( -\frac{1}{4\pi ^{2}\left( \left( 16.09\unit{kHz}\right) ^{3}\right)
\left( 26.4\times 10^{-9}\unit{F}\right) }\left( \allowbreak \allowbreak
0.1\allowbreak 609\unit{kHz}\right) \right) ^{2}=\allowbreak \frac{%
1.\,\allowbreak 373\,6\times 10^{-9}}{\unit{A}^{4}}\frac{\unit{m}^{4}}{\unit{%
s}^{4}}\unit{kg}^{2}$

\bigskip $\left( -\frac{1}{4\pi ^{2}\left( \left( 16.09\unit{kHz}\right)
^{2}\right) \left( 26.4\times 10^{-9}\unit{F}\right) ^{2}}\left( \allowbreak
5.\,\allowbreak 28\times 10^{-9}\unit{F}\right) \right) ^{2}=\allowbreak 
\frac{5.\,\allowbreak 494\,3\times 10^{-7}}{\unit{A}^{4}}\frac{\unit{m}^{4}}{%
\unit{s}^{4}}\unit{kg}^{2}$

$\delta L_{A}=\sqrt{\left( -\frac{1}{4\pi ^{2}\left( \left( 16.09\unit{kHz}%
\right) ^{3}\right) \left( 26.4\times 10^{-9}\unit{F}\right) }\left(
\allowbreak 0.1\allowbreak 609\unit{kHz}\right) \right) ^{2}+\left( -\frac{1%
}{4\pi ^{2}\left( \left( 16.09\unit{kHz}\right) ^{2}\right) \left(
26.4\times 10^{-9}\unit{F}\right) ^{2}}\left( \allowbreak 5.\,\allowbreak
28\times 10^{-9}\unit{F}\right) \right) ^{2}}=\allowbreak \frac{%
7.\,\allowbreak 421\,6\times 10^{-4}}{\unit{A}^{2}}\frac{\unit{m}^{2}}{\unit{%
s}^{2}}\unit{kg}$

Then from our resonant frequency measurement we have

$L_{A}=\allowbreak \allowbreak \frac{3.\,\allowbreak 706\,2\times 10^{-3}}{%
\unit{A}^{2}}\frac{\unit{m}^{2}}{\unit{s}^{2}}\unit{kg}\pm \frac{%
7.\,\allowbreak 421\,6\times 10^{-4}}{\unit{A}^{2}}\frac{\unit{m}^{2}}{\unit{%
s}^{2}}\unit{kg}$ 

and we need to compare this to our theoretical value from our inductance
equation

$L=\left( \allowbreak \frac{3.\,\allowbreak 676\,2\times 10^{-3}}{\unit{A}%
^{2}}\frac{\unit{m}^{2}}{\unit{s}^{2}}\unit{kg}\right) \pm (\frac{%
1.\,\allowbreak 638\,7\times 10^{-3}}{\unit{A}^{2}}\frac{\unit{m}^{2}}{\unit{%
s}^{2}}\unit{kg})$ 

It looks like these are pretty close. A quick graph makes comparing them easy

Points

$%
\begin{array}{cc}
1 & 3.\,\allowbreak 706\,2\times 10^{-3} \\ 
2 & 3.\,\allowbreak 676\,2\times 10^{-3}%
\end{array}%
$

Error Bars

$%
\begin{array}{cc}
1 & 3.\,\allowbreak 706\,2\times 10^{-3}-7.\,\allowbreak 421\,6\times 10^{-4}
\\ 
1 & 3.\,\allowbreak 706\,2\times 10^{-3}+7.\,\allowbreak 421\,6\times 10^{-4}%
\end{array}%
$

$%
\begin{array}{cc}
2 & 3.\,\allowbreak 676\,2\times 10^{-3}+1.\,\allowbreak 638\,7\times 10^{-3}
\\ 
2 & 3.\,\allowbreak 676\,2\times 10^{-3}-1.\,\allowbreak 638\,7\times 10^{-3}%
\end{array}%
$

\FRAME{dtbpFX}{2.4232in}{3in}{0pt}{}{}{Plot}{\special{language "Scientific
Word";type "MAPLEPLOT";width 2.4232in;height 3in;depth 0pt;display
"USEDEF";plot_snapshots TRUE;mustRecompute FALSE;lastEngine "MuPAD";xmin
"0";xmax "2";xviewmin "0.9999";xviewmax "2.0001";yviewmin "0.002";yviewmax
"0.00531522774";rangeset"X";plottype 4;plotticks 1;num-x-ticks
1;labeloverrides 3;axesFont "Times New
Roman,12,0000000000,useDefault,normal";numpoints 100;plotstyle
"patchnogrid";axesstyle "normal";axestips FALSE;xis \TEXUX{x};var1name
\TEXUX{$x$};function
\TEXUX{$\MATRIX{2,2}{c}\VR{,,c,,,}{,,c,,,}{,,,,,}\HR{,,}\CELL{1}\CELL{3.\,%
\allowbreak 706\,2\times 10^{-3}}\CELL{2}\CELL{3.\,\allowbreak 676\,2\times
10^{-3}}$};linecolor "black";linestyle 1;pointplot TRUE;pointstyle
"point";linethickness 3;lineAttributes "Solid";curveColor
"[flat::RGB:0000000000]";curveStyle "Point";function
\TEXUX{$\MATRIX{2,2}{c}\VR{,,c,,,}{,,c,,,}{,,,,,}\HR{,,}\CELL{1}\CELL{3.\,%
\allowbreak 706\,2\times 10^{-3}-7.\,\allowbreak 421\,6\times
10^{-4}}\CELL{1}\CELL{3.\,\allowbreak 706\,2\times 10^{-3}+7.\,\allowbreak
421\,6\times 10^{-4}}$};linecolor "blue";linestyle 1;pointstyle
"point";linethickness 3;lineAttributes "Solid";curveColor
"[flat::RGB:0x000000ff]";curveStyle "Line";function
\TEXUX{$\MATRIX{2,2}{c}\VR{,,c,,,}{,,c,,,}{,,,,,}\HR{,,}\CELL{2}\CELL{3.\,%
\allowbreak 676\,2\times 10^{-3}+1.\,\allowbreak 638\,7\times
10^{-3}}\CELL{2}\CELL{3.\,\allowbreak 676\,2\times 10^{-3}-1.\,\allowbreak
638\,7\times 10^{-3}}$};linecolor "blue";linestyle 1;pointstyle
"point";linethickness 3;lineAttributes "Solid";curveColor
"[flat::RGB:0x000000ff]";curveStyle "Line";VCamFile
'RVWGF512.xvz';valid_file "T";tempfilename
'RVWGF508.wmf';tempfile-properties "XPR";}}

The uncertainty ranges overlap! This looks good. I have not falsified our
inductance model. There is some concern about the number of layers in the
inductor. But within what I can tell without taking apart the coil this
seems reasonable. 

\end{document}
